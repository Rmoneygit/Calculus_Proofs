\documentclass{article}
\usepackage[utf8]{inputenc}
\usepackage{amsmath, amssymb, bm}
\usepackage{gensymb}

\begin{document}
	\textbf{Theorem 1.4.6:} Let $f$ be a real-valued function and $a$, $L$ be real numbers and $n > 0$ be an integer. Suppose
	$$\lim_{x \to a}{f(x)} = L$$
	Then
	$$\lim_{x \to a}[{f(x)^n}] = L^n$$
	\textbf{Proof:}
	We will proceed by induction. \\\\
	\textit{Step 1: Base case} \\
	Let $n = 1.$ Then
	\begin{align*}
		\lim_{x \to a}[{f(x)^n}] &= \lim_{x \to a}[{f(x)^1}] \\
		                         &= \lim_{x \to a}{f(x)} \\
		                         &= L \\
		                         &= L^1 \\
		                         &= L^n.
	\end{align*}
	So the statement holds for the base case. \\\\
	\textit{Step 2: Induction step} \\
	Suppose $\lim_{x \to a}[{f(x)^n}] = L^n$. We then need to show that
	$$\lim_{x \to a}[{f(x)^{n+1}}] = L^{n+1}$$
	We have
	\begin{align*}
		\lim_{x \to a}[{f(x)^{n+1}}] &= \lim_{x \to a}[{f(x)^nf(x)}] \\
		                             &= \lim_{x \to a}{f(x)^n} * \lim_{x \to a}{f(x)} \tag{Theorem 1.4.2} \\
		                             &= L^n * L \\
		                             &= L^{n+1}
	\end{align*}
	which is what we needed to show. \\\\
	We have proven the base case and the induction step. Thus, by the principle of mathematical induction, the statement 
	$$\lim_{x \to a}[{f(x)^n}] = L^n$$
	holds for all positive integers $n$.
\end{document}