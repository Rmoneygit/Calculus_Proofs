\documentclass{article}
\usepackage[utf8]{inputenc}
\usepackage{amsmath, amssymb, bm}
\usepackage{gensymb}

\begin{document}
	\textbf{Theorem 1.4.5:} Let $f$ and $g$ be real-valued functions and $a$, $L$ and $M$ be real numbers with $M \neq 0$. Suppose that
	$$\lim_{x \to a}{f(x)} = L \; \text{ and } \lim_{x \to a}{g(x)} = M$$
	Then
	$$\lim_{x \to a}{\frac{f(x)}{g(x)}} = \frac{L}{M}.$$
	\textbf{Proof:}
	To show this, let us first show that
	$$\lim_{x \to a}{\frac{1}{g(x)}} = \frac{1}{M}$$
	Let $\epsilon > 0$ be an arbitrary real number. We then need to show that there exists some real number $\delta > 0$ such that
	$$0 < |x - a| < \delta \implies \left|\frac{1}{g(x)} - \frac{1}{M}\right| < \epsilon$$
	\textit{Part 1: Choose a $\delta$:} \\
	Note that
	\begin{align*}
		\left|\frac{1}{g(x)} - \frac{1}{M}\right| &= \left|\frac{1}{g(x)}\frac{M}{M} - \frac{1}{M}\frac{g(x)}{g(x)}\right| \tag{$M \neq 0 \implies g(x) \neq 0$} \\
		                               &= \left|\frac{M - g(x)}{Mg(x)} \right| \\
		                               &= \frac{|M - g(x)|}{|Mg(x)|}
	\end{align*}
	We need to choose a $\delta$ that makes this expression less than $\epsilon$. Since $\lim_{x \to a}{g(x)} = M$ and $\frac{|M|}{2} > 0$, then there exists a real number $\delta_{1} > 0$ such that
	$$0 < |x - a| < \delta_{1} \implies |g(x) - M| < \frac{|M|}{2}$$
	and therefore
	\begin{align*}
		|M| &= |M - g(x) + g(x)| \\
		    &\leq |M - g(x)| + |g(x)| \tag{Triangle Ineq.} \\
		    &= |-(-M + g(x))| + |g(x)| \\
		    &= |g(x) - M| + |g(x)| \\
		    &< \frac{|M|}{2} + |g(x)|
	\end{align*}
	which implies that
	\begin{align*}
		|M| &< \frac{|M|}{2} + |g(x)| \\
		2|M| &< |M| + 2|g(x)| \\
		|M| &< 2|g(x)| \\
		\frac{|M|}{2} &< |g(x)|
	\end{align*}
	So we have shown that
	\begin{align*}
	0 < |x - a| < \delta_{1} &\implies \frac{|M|}{2} < |g(x)| \\\\
	                         &\implies \frac{2}{|M|} > \frac{1}{|g(x)|}
	\end{align*}
	and so, for these values of $x$,
	\begin{align*}
		\frac{1}{|Mg(x)|} &= \frac{1}{|M||g(x)|} \\
		                  &= \frac{1}{|M|} * \frac{1}{|g(x)|} \\
		                  &< \frac{1}{|M|} * \frac{2}{|M|} \\
		                  &< \frac{2}{|M||M|} \\
		                  &< \frac{2}{|M^2|} \\
		                  &< \frac{2}{M^2} \tag{$M^2 > 0$}
	\end{align*}
	Since $\lim_{x \to a}{g(x)} = M$ and $\frac{M^2}{2}\epsilon > 0$, then there exists a real number $\delta_{2} > 0$ such that
$$0 < |x - a| < \delta_{2} \implies |g(x) - M| < \frac{M^2}{2}\epsilon$$
This suggests we should choose $\delta = \min{(\delta_{1}, \delta_{2})}$. \\\\
\textit{Step 2: Verify $\delta$ satisfies the condition} \\
Suppose $0 < |x - a| < \delta$. This implies that
$$0 < |x - a| < \delta_{1} \; \text{ and } \; 0 < |x - a| < \delta_{2}$$
These imply that
$$\frac{1}{|Mg(x)|} < \frac{2}{M^2} \; \text{ and } \; |g(x) - M| < \frac{M^2}{2}\epsilon$$
Positive inequalities multiply, so we can multiply these to obtain
\begin{align*}
	\frac{|g(x) - M|}{|Mg(x)|} &< \frac{2}{M^2}\frac{M^2}{2}\epsilon \\
	\left|\frac{1}{g(x)} - \frac{1}{M}\right| &< \epsilon
\end{align*}
which is what we needed to show. So $\delta$ exists and it satisfies the condition. Thus
$$\lim_{x \to a}{\frac{1}{g(x)}} = \frac{1}{M}$$
Finally, use this result and Theorem 1.4.2 to show
\begin{align*}
	\lim_{x \to a}{\frac{f(x)}{g(x)}} &= \lim_{x \to a}{\left[f(x)\frac{1}{g(x)}\right]} \\
	                                  &= \lim_{x \to a}{f(x)} * \lim_{x \to a}{\frac{1}{g(x)}} \\
	                                  &= L * \frac{1}{M} \\
	                                  &= \frac{L}{M}
\end{align*}
which is what we wanted to show.
\end{document}